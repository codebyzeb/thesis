\chapter{Introduction}\label{chapter:intro}

The field of Natural Language Processing (NLP), which is concerned with the computational analysis, generation and processing of natural language, has undergone several paradigms shifts during its history. The current paradigm is dominated by the use of \textbf{Deep Neural Networks} (DNNs) to learn general-purpose representations of language that transfer well to downstream tasks. The representations are learned using self-supervised tasks that allow the neural network to learn \textbf{contextual embeddings} for language split into discrete \textbf{tokens}. This learning process is called \textbf{pre-training} and the resulting networks are referred to as \textbf{Pretrained Language Models} (PLMs), or often simply as \textbf{Language Models} (LMs). 

This pre-training process is almost entirely end-to-end, except for the very first step: the conversion of language into discrete tokens, which in this thesis is referred to as the \textbf{input representation}. In some of the earliest examples of neural networks being used to study the sequential properties of language, these tokens may have consisted of individual characters or even \textbf{phonemes}, allowing researchers to theorize about human language learning and processing. The input representation now used to train LMs is not motivated by linguistics or cognition, but by efficiency. A major benefit of the pre-training process is that it is far more scalable than previous methods in NLP, not requiring explicitly labelled data. Now, the largest of these networks, known as \textbf{Large Language Models} (LLMs) consist of billions of parameters, are trained on trillions of words, and are remarkably powerful across a broad range of tasks and applications. As a result of this success, the NLP has converged on an input representation that make this scaling feasible. As the easiest means of sourcing trillions of words of data is web-scraping \addcites, tokens predominantly represent \textbf{orthographic} text\footnote{Web-scraping also means that LLMs are predominantly trained on English...}. Instead of characters or words, the discrete unit used to split this data is the \textbf{subword}, an arbitrary unit which typically lies between between characters and words in length but is not motivated by morphology. Instead, subword tokenisers such as BPE seem to seek a balance between many (often competing) technical constraints, such as compression rate, vocabulary size and the handling of out-of-vocabulary (OOV) items.

This \textbf{default input representation} has become ubiquitous in NLP and although many alternatives to popular subword tokenisation algorithms have been proposed, there have been relatively few studies challenging this ubiquity. One alternative representation consists of individual \emph{phonemes}, the perceptually distinct units of sound in a specified language used to distinguish words in speech. Instead of orthographic subwords, the \textbf{phonemic input representation} consists of individual phonemes, often without word boundaries, as these are not easily available in speech. Although a few studies have explored specific practical and analytical benefits of using this representation to train language models (see \cref{sec:12-phoneval} for a review), phoneme-level modelling is still vastly under-studied. There is a clear need for a thorough examination of this representation in order to identify possible resource constraints %that could be preventing work in this area,
and to establish the feasibility of using the phonemic input representation to train modern LM architectures.% through a comparison with the standard input representation.

%there are limited insights that can be gained when relating these models to phonology (as the representation is orthographic) or human language learning. %Phoneme-level modelling is vastly under-studied, despite the potential benefits for phonological analysis, alternative tokenisation, multilingual modelling, and low-resource modelling. 

The phoneme input representation could also have significant implications for research using LMs to investigate the distributional properties of language, updating legacy `connectionist' models (which often did use phonemes) with much more powerful architectures. In the current paradigm, LMs are frequently compared to human learners; however, this comparison is complicated by the fact that models are typically trained on vastly more linguistic data than humans encounter over their lifetimes \citep{huebner-etal-2021-babyberta}. This concern has motivated the \emph{BabyLM challenge} \citep{warstadt-2023-babylm-findings,conll-2024-babylm}, where so called `BabyLMs' are pre-trained using corpora whose scale and source are considered ``developmentally plausible''. Despite the aim to improve comparisons between language models and human learners, the default input representation has become so entrenched that it is not even challenged here, despite the divergence from the auditory input available to human infants. Reevaluating the input modality is thus a crucial step toward more realistic comparisons with human language acquisition.\zeb{Potentially needs rephrasing to be more targeted towards general study of phonology as done in \cref{chapter:phonology}}

%The choice of input representation—such as phonemes—has significant implications for research using language models (LMs) to investigate the distributional properties of language, particularly as modern architectures build on the legacy of early connectionist models. In the current landscape, LMs are frequently compared to human learners; however, this comparison is complicated by the fact that models are typically trained on vastly more linguistic data than humans encounter over their lifetimes \citep{huebner-etal-2021-babyberta}. This concern has motivated the \emph{BabyLM Challenge} \citep{warstadt-2023-babylm-findings,findings,conll-2024-babylm}, which promotes the development of LMs trained on data that is both limited in scale and developmentally plausible. Despite this promising direction, most BabyLMs still rely on standard input formats—typically text segmented into subword units—even though such representations diverge significantly from the auditory, phonemic input available to human infants. Reevaluating the input modality is thus a crucial step toward more realistic models of human language acquisition.

% Particularly easy to scale compared to past NLP methods since only unlabelled text is required. Instead,  Benefiting from the vast quantities of text available online through procedures like web-scraping, state-of-the-art has been dominated by scaling this pre-training step as much as possible, using particularly large networks trained on especially massive corpora, aptly-named Large Language Models. 


%Standard NLP pipeline uses large data, orthographic, scraped from the web, with subword tokenisation for efficiency and downstream performance. (+high level explanation here) Remarkable power and almost entirely end-to-end, save for a crucial component: the tokeniser, which is often trained in advance and uses a subword representation motivated entirely by efficiency, rather than being motivated by cognition or linguistic boundaries. Call this the ``standard input representation''.

% Possibly for background instead: Now, pre-training is so expensive that very large models are only trained once, and later fine-tuned for specific tasks, however since the tokeniser is `locked-in', this often limits out-of-domain fine-tuning due to the subword vocabulary badly representing the new task. In some cases, pre-training a smaller network is more effective and in fact less computationally expensive than fine-tuning a large network.

%The earliest `LMs' (then called "connectionist" networks) used a completely different input representation - often individual phonemes or characters, in order to best model the brain's processing of language by choosing appropriate units for modelling. These networks were not built to handle noisy data and so `tokeniser' may not be the correct term. Modern LLMs continue to be a useful tool for linguistic analysis. In response to the astounding performance of transformer-based architectures on downstream tasks, one areas of research investigated the linguistic capabilities of these models....

%However, when trained on internet-scraped written text using a subword-based tokeniser, there are limited insights that can be gained when relating these models to phonology (as the representation is orthographic) or human language learning. In recent years, the unrealistic scale of the pre-training data used in the standard NLP pipeline has led to an alternative direction of research, where models trained on developmentally plausible data (not just in scale, but also in domain) have led to new insights... BabyLM challenge... particularly as these experiments are feasible with academic budgets.

%Despite the focus on developmentally plausible data in the BabyLM research community, the subword input representation is so entrenched that the vast majority of entries continued to use subword tokenisers. This is despite the fact that subword tokenisation is not cognitively plausible...etc...

%Phoneme-level modelling is vastly under-studied, despite the potential benefits for phonological analysis, alternative tokenisation, multilingual modelling, and low-resource modelling. 

The focus of this thesis is to establish the benefits and insights that can be gained from using a phonemic input representation to train modern NLP architectures, particularly when considering developmentally plausible training data. First, finding a lack of suitable tools and datasets for phonemic language modelling, I develop and release two open-source resources which also support phonological analysis more broadly. I then carry out a wide range of experiments leveraging these resources in order to assert the feasibility of the phonemic input representation for training transformer-based architectures. These experiments also establish how phonological and syntactic capabilities scale with data and model size. Using these findings and resources, I demonstrate how the phonemic input representation supports phonological analysis by training LMs on child-directed speech across 31 languages and probing the trained models for phonological features. Using a word segmentation task drawn from the acquisition literature, I find that these models implicitly track word boundaries, despite word boundaries not appearing during training. Finally, noting the parallels between the word segmentation task and bottom-up subword tokenisers, I propose a novel method for learning token merges and find that \writemore


\Zeb{maybe end with something like... demonstrating that research into cognitively-motivated modelling can lead to new improvements even in industry-scale use of these models...}

% 6. I address these questions in \cref{chapter:resources} and \cref{chapter:modelling}. 



% LMs can also be a useful tool for linguistic analysis, studying learning dynamics, and drawing parallels to acquisition using developmentally plausible data. 




% 3. As large-scale language modelling has grown out-of-reach for academic budgets, such explorations have developed as an alternative research direction (BabyLM).

%3. The earliest examples of auto-regressive neural networks were developed for linguistic analysis, not for downstream use. 


%3. Phoneme-level modelling is under-studied, despite the benefits for phonological analysis, alternative tokenisation, multilingual modelling, and low-resource modelling. Resources exist but are limited in availability and few are cross-lingual.

%4. Hence, we must introduce new resources (chapter 3) and properly determine if LM architectures are suitable for modelling with phonemes (chapter 4).

%5. We can then model cross-lingually and probe these models for phonological knowledge using word segmentation (chapter 5). 

%6. Finally, word segmentation could inspire new methods for tokenisation in the standard NLP setup (chapter 6), demonstrating the downstream benefits of research that seeks to explore language and language learning more broadly...

\section{Research questions and contributions}

The research conducted in this thesis investigates the feasibility of using phonemes as input to modern LLM architectures and whether this input representation can lead to new insights across phonological experimentation and tokenisation. The notion of feasibility can be broken down into two research questions:

\begin{question}{}{resources}
    To what extent do sufficient resources exist for training and evaluating language models using a phonemic input representation?
\end{question}

\begin{question}{}{modelling}
    To what extent are transformer-based architectures suitable for pre-training with the phonemic input representation?
\end{question}

I address these two questions through the following targeted contributions:

\begin{itemize}
    \item I conduct a thorough analysis of existing datasets and grapheme-to-phoneme tools and find crucial gaps in existing resources.
    \item I develop and release two resources that fill these gaps; the \ipachildes dataset and the \gpp tool for grapheme-to-phoneme conversion.
    \item Using the BabyLM challenge framework, I compare models trained with a phonemic input representation to models trained with a default subword input representation by considering three key transformations.
    \item Using the North American English section of \ipachildes, I compare the effect of model size and data size on syntactic and phonological capabilities of phoneme models. 
\end{itemize}

Addressing these research questions facilitates the use of these models to seek new insights. With these newly developed resources and the suitability of transformer-based architectures for this research established, \cref{chapter:phonology,chapter:infotokenisation} tackle the following two research questions:

\begin{question}{}{phonology}
    To what extent does the phonemic input representation facilitate cross-lingual phonological experimentation and analysis?
\end{question}

\begin{question}{}{tokenisation}
    To what extent can phonological experimentation with language models lead to the development of alternative algorithms for tokenisation? 
\end{question}

I contribute to these questions through the following specific contributions:

\begin{itemize}
    \item I probe models trained on 11 languages from \ipachildes for phonological distinctive features, possible due to the coherence that \gpp maintains with a pre-existing phonological database. 
    \item I investigate whether phoneme LMs implicitly track word boundaries by using the word segmentation task, proposing a novel unsupervised method based on models in the acquisition literature which I use to use alongside linear probes to evaluate models across 31 languages.
    \item Noting the parallels between information-driven word segmentation strategies and subword tokenisation, I propose and evaluate a novel tokenisation strategy that bases merges on surprisal, rather than frequency.
\end{itemize}

I first note the lack of suitable tools and datasets for phonemic language modelling and so develop and release two open-source resources; an improved grapheme-to-phoneme tool for converting orthographic text to a phonemic representation and a phonemic dataset of child-centred speech across 31 languages. I then use these resources to explore the feasibility of phoneme-based pre-training of modern NLP architectures. Noting the three transformations that separate the phonemic input representation from the default input representation, I use linguistic judgement tasks and downstream tests to both confirm this feasibility and establish the size requirements for phoneme-based language modelling across data and model sizes. 

Using linguistic judgement tasks, I assess the data and size requirements for phoneme-based there are few phonemic datasets large enough for training LLMs and none that are both multilingual and naturalistic, and tools for converting orthographic text to a phonemic representation 


First, we must establish whether suitable resources are available to facilitate this work, and whether modern transformer-based architectures support a phonemic input representation. This can be split into two research questions:

\section{Thesis outline}

A summary of the subsequent chapters of this thesis is given below.

\begin{itemize}
    \item In \cref{chapter:background}, I provide the necessary background for the research presented in this thesis. I discuss the role of tokenisation in modern NLP architectures and define the phonemic input representation in relation to the default input representation by finding three transformations that separate them, noting previous studies that have examined these transformations. I then examine prior work that has used phoneme LMs for cross-lingual phonological experimentation, studying theories of acquisition and downstream applications, finding relatively few benchmarks for phonological probing. Finally, I discuss research surrounding `BabyLMs', smaller models pre-trained on developmentally plausible corpora, which have led to numerous advancements and insights but have predominantly focused on English and mostly used a subword representation, despite the relative implausibility of orthographic text split into subwords, which can be addressed by the phonemic input representation.
    \item In \cref{chapter:resources}, I conduct a thorough review of existing phonemic datasets and methods used for creating these datasets. I identify a particular lack of data for cross-lingual work to do with spontaneous or child-centred speech. I find that the majority of large datasets are created using grapheme-to-phoneme conversion tools but that these tools lack coherence with established phonemic inventories and phonological databases. I address both of these issues by developing two resources which are released open-source; \gpp, an improved grapheme-to-phoneme conversion tool which ensures that phoneme inventories correspond to inventories from an established phonological database, and \ipachildes, a phonemic dataset of child-centred, spontaneous speech across 31 languages, created using \gpp.
    \item In \cref{chapter:modelling}, I explore the feasibility of training LLMs with a phonemic input representation. I begin by defining model architecture used for the experimental work in this thesis. I then present eight tokenisers that cover all combinations of the three input transformations and compare models trained with these tokenisers using the BabyLM challenge dataset and evaluation pipeline, finding that phoneme LMs only lead to a minor decrease in performance on these traditional benchmarks but are beneficial for phonological benchmarks. I then establish the scaling laws of phoneme LMs by training models of various sizes on subsets of North American English portion of \ipachildes and evaluating their syntactic and phonological capabilities.
    \item In \cref{chapter:phonology}, having established appropriate model sizes for the varying availabilities of data in \ipachildes, I train and release a suite of phoneme LMs across 31 languages to support phonological experimentation. I demonstrate how the correspondence that \gpp maintains with a phonological database allows for these models to be probed for implicit knowledge of distinctive features and find that the distributional properties of phonemes are sufficient to learn major class and place features cross-lingually. I then turn to the word segmentation task from the acquisition literature, proposing an unsupervised method for extracting word boundaries from phoneme LMs as well as using linear probes, finding that these models implicitly track word boundaries and noting priors inherit in the distributional properties of phonemes in different languages.
    \item In \cref{chapter:infotokenisation}, I note the similarities between my unsupervised approach to word segmentation and alternative methods for the tokenisation step in LLMs. Returning to the tokenisers compared in \cref{chapter:modelling}, I propose an alternative method for learning phoneme-based subwords using the predictions of a phoneme LM and find that this leads to improvements across syntactic and phonological tasks. Applying this strategy to a standard language modelling setup, whereby byte merges are based on predictions from a byte-level LM rather than frequency, I find\ldots \zeb{needs completing}. 
    \item In \cref{chapter:conclusion}, I summarise the contributions of this thesis and discuss the wider implications. I propose ideas for future directions and ending with some concluding remarks.
\end{itemize}

\section{Publications}

The research presented in this thesis has been the basis for the following publications.

\begin{itemize}
\item The work presented in \cref{chapter:resources}, the experiments using the IPA-CHILDES dataset in \cref{chapter:modelling} and the phonological feature probing experiments in \cref{chapter:phonology} are published as:

\begin{mdframed}[linewidth=1pt]
Zébulon Goriely and Paula Buttery. 2025. \href{https://arxiv.org/abs/2504.03036}{IPA-CHILDES \& G2P+: Feature-Rich Resources for Cross-Lingual Phonology and Phonemic Language modelling.} Currently under review at CoNLL 2025.
\end{mdframed}

\item The work presented in \cref{chapter:modelling} (excluding the experiments using the IPA-CHILDES dataset) won the ``Outstanding Paper'' award at the second edition of the BabyLM challenge and is published as:

\begin{mdframed}[linewidth=1pt]
Zébulon Goriely, Richard Diehl Martinez, Andrew Caines, Paula Buttery, and Lisa Beinborn. 2024. \href{https://aclanthology.org/2024.conll-babylm.4/#}{From Babble to Words: Pre-Training Language Models on Continuous Streams of Phonemes}. In The 2nd BabyLM Challenge at the 28th Conference on Computational Natural Language Learning, pages 37–53, Miami, FL, USA. Association for Computational Linguistics.
\end{mdframed}

\item The work presented in \cref{chapter:phonology} (excluding the phonological feature probing experiments) is published as:
\begin{mdframed}[linewidth=1pt]
Zébulon Goriely and Paula Buttery. 2025. \href{https://arxiv.org/abs/2504.03338}{BabyLM's First Words: Word Segmentation as a Phonological Probing Task}. Currently under review at CoNLL 2025.
\end{mdframed}

\item The work presented in \cref{chapter:infotokenisation} is published as:
\begin{mdframed}[linewidth=1pt]
Insert paper here.
\end{mdframed}
\end{itemize}

Additionally, several related publications were produced alongside the papers listed above. These are listed below due to being highly related to this research but are not included in this thesis.

\begin{itemize}
\item The winner of the \enquote{Most Interesting Paper} award at the first edition of the BabyLM challenge, exploring the use of curriculum learning for training LLMs in a developmentally plausible setting:
\begin{mdframed}[linewidth=1pt]
    Richard Diehl Martinez, Zébulon Goriely, Hope McGovern, Christopher Davis, Andrew Caines, Paula Buttery, and Lisa Beinborn. 2023. \href{https://aclanthology.org/2023.conll-babylm.10/}{CLIMB – Curriculum Learning for Infant-inspired Model Building}. In Proceedings of the BabyLM Challenge at the 27th Conference on Computational Natural Language Learning, pages 112–127, Singapore. Association for Computational Linguistics.
\end{mdframed}

\item Introducing a measure of \enquote{Frequency Bias} in LLMs and introducing \enquote{Syntactic Smoothing}, using a prior based on part-of-speech classes during training, to mitigate this bias:
\begin{mdframed}[linewidth=1pt]
    Zébulon Goriely, Richard Diehl Martinez, Andrew Caines, Paula Buttery, and Lisa Beinborn. 2024. \href{https://aclanthology.org/2024.emnlp-main.344/}{Mitigating Frequency Bias and Anisotropy in Language Model Pre-Training with Syntactic Smoothing}. In Proceedings of the 2024 Conference on Empirical Methods in Natural Language Processing, pages 5999–6011, Miami, Florida, USA. Association for Computational Linguistics.
\end{mdframed}

\item Exploring the use of fine-grained curriculum learning strategies based on linguistic acquisition theories across several languages:
\begin{mdframed}[linewidth=1pt]
    Suchir Salhan, Richard Diehl Martinez, Zébulon Goriely, and Paula Buttery. 2024. \href{https://aclanthology.org/2024.conll-babylm.15/}{Less is More: Pre-Training Cross-Lingual Small-Scale Language Models with Cognitively-Plausible Curriculum Learning Strategies}. In The 2nd BabyLM Challenge at the 28th Conference on Computational Natural Language Learning, pages 174–188, Miami, FL, USA. Association for Computational Linguistics.
\end{mdframed}

\item My master's thesis exploring a cue-based algorithm for word segmentation, which was augmented with additional experiments:
\begin{mdframed}[linewidth=1pt]
    Zébulon Goriely, Andrew Caines and Zébulon Goriely. 2023. \href{https://www.cambridge.org/core/journals/journal-of-child-language/article/word-segmentation-from-transcriptions-of-childdirected-speech-using-lexical-and-sublexical-cues/5D206536C878ADA80026ED8E2E72F848}{Word segmentation from transcriptions of child-directed speech using lexical and sub-lexical cues}. Journal of Child Language. 2025;52(1):1-41.
\end{mdframed}
\end{itemize}

\section{Open-source contributions and research artefacts}

The two resources presented in \cref{chapter:resources} have been released as open-source:

\begin{itemize}
\item \gpp is a pip-installable tool using an MIT license. The source is available on \href{https://github.com/codebyzeb/g2p-plus}{GitHub}. 
\item The \ipachildes dataset is hosted on \href{https://huggingface.co/datasets/phonemetransformers/IPA-CHILDES}{Huggingface}.
\end{itemize}

Additionally, the following research artefacts have been released for reproducibility and dissemination:

\begin{itemize}
    \item \href{https://github.com/codebyzeb/childes-processor}{https://github.com/codebyzeb/childes-processor}: The source code for \childesprocessor, a pip-installable tool used to download and process the CHILDES database for the creation of the \ipachildes (\cref{chapter:resources}). It is open-sourced with an MIT license.
    \item \href{https://github.com/codebyzeb/babylm-ipa}{https://github.com/codebyzeb/babylm-ipa}: The scripts used for leveraging \gpp to convert the BabyLM training data and evaluation data to an IPA representation (\cref{chapter:modelling}).
    \item \href{https://github.com/codebyzeb/PhonemeTransformers}{https://github.com/codebyzeb/PhonemeTransformers}: The source code and scripts used to train language models using phonemes, compare input representations and evaluate models using word segmentation and phonological features probes (\cref{chapter:modelling,chapter:phonology}). 
    \item \href{https://huggingface.co/phonemetransformers}{https://huggingface.co/phonemetransformers}: A Huggingface organisation containing all datasets, models and tokenisers trained for the experiments in \cref{chapter:modelling,chapter:phonology}, grouped into collections for convenience.
    \item \href{https://github.com/codebyzeb/infotokenization}{https://github.com/codebyzeb/infotokenization}: The scripts used to train byte-level language models, use their predictions to create subword tokenisers, train subword-level models using these tokenisers and evaluate them (\cref{chapter:infotokenisation}).
    \item \href{https://huggingface.co/Infotokenizers}{https://huggingface.co/Infotokenizers}: A Huggingface organisation containing all datasets, models and tokenisers trained for the experiments in \cref{chapter:infotokenisation}.
\end{itemize}

% \section{Statement of style}

% \Zeb{Use of ``we'' in the thesis for stylistic reasons and to acknowledge the contribution of co-authors. Mixed spelling between British English and American English due to my mixed background and the dominance of American spelling in technical terminology. British English is used throughout, except for the word "tokeniser". }
