\chapter{Background}\label{chapter:background}

\Zeb{Argument of thesis is that it is important to explore the phonemic representation. In \cref{sec:12-tokenization} I define ``input representation'', establishing a contrast between the default and phoneme input representation. Review past work on input representation in language models and recent work on tokenisation methods, concluding that the phoneme input representation is largely under-studied in the modern NLP landscape. In \cref{sec:12-phoneval} I review work regarding phonological evaluation of language models. This includes the use of the phoneme input representation in early connectionist models of language processing and computational models of acquisition as well as more recent work using phonemes in language models to study phonology cross-lingually to benchmarks that directly probe the phonological capabilities of LMs. Finally, in \cref{sec:12-babylm} I review work concerning the use of developmentally-plausible language models. Such models provide a useful framework for studying linguistic theories but mostly continue to use the default input representation, despite children not learning from arbitrary subwords. This chapter concludes that there is a clear need to study the use of the phoneme input representation in modern LM architectures.}

\section{Input representations in language models}\label{sec:12-tokenization}

\Zeb{Formally define what I mean by input representation. Give the broad definition of tokenizer without going into too much detail. Discuss orthographic text more.}

\begin{figure}[t]
    \centering
    \includegraphics[width=0.99\linewidth]{example-image-a}
    \caption{The standard classical and modern pipeline for preparing text for language modelling tasks.}
    \label{fig:12-pipelinecomparison}
\end{figure}

\textbf{Language models} are distributional models of natural language; instead of using a grammar to determine the structure of a sentence, they provide a probability distribution for each lexical item based on its context --- the ``company it keeps'' \citep{firth1957synopsis}. These lexical items are typically discrete items called \textbf{tokens} drawn from a fixed vocabulary \(V\). For a sequence of tokens \(w_1, w_2, \dots, w_T\), the language model estimates the probability:

\[
P(w_1, w_2, \dots, w_T) = \prod_{t=1}^{T} P(w_t \mid w_1, w_2, \dots, w_{t - 1})
\]

\[
P(w_1, w_2, \dots, w_T) = \prod_{t=1}^{T} P\left(w_t \mid w_1, w_2, \dots, w_{t-1}\right)
\]

\begin{align}
    P(w_1, w_2, \dots, w_T) &= \prod_{t=1}^{T} P\left(w_t \mid w_1, w_2, \dots, w_{t-1}\right)
\end{align}


Definition of language models - that they predict units given a history of previous units. Early language models were n-grams, which often operated with individual characters or words. This was fine theoretically, but in practice, text is noisy, containing many non-lexical artifacts like numbers and punctuation. What should constitute a ``word'' is also debated, e.g. should clitics like `'s' be a separate token, should compound words like ``bad-mouth'' be split? Handling this complicated task was traditionally one of the first tasks of preparing written text for NLP tasks. Preceding steps could include by text-cleaning operations like lowercasing and the removeal of unwanted symbols like punctuation, with post-tokenisation steps including stemming, part-of-speech tagging and stop-word removal depending on the task. This pipeline is illustrated in \cref{fig:12-pipelinecomparison} and is offered by python packages like NLTK \addcites.

During the shift from count-based representations to neural representations like Word2Vec \addcites, there was also a shift in how words were tokenized. Word-based tokenization suffered from very large vocabulary sizes and out-of-vocabulary (OOV) issues, often using UNK tokens to handle these. An alternative is to split unknown words into subwords, falling back to character n-grams, as pioneered by the FastText embedding model \addcites. This would not be a good idea for n-gram models since more tokens decreases the context seen a fixed context window and increases the vocabulary size further, but embedding models were improved to handle larger, variable context sizes. LSTMs providing a major improvement over the vanishing gradients of RNNs and were used in \writemore models. Later, the transformer architecture improved memory further using the attention mechanism \addcites.

Vocabulary size was still a major issue, linearly scaling the size of the embedding layer (often containing a large percentage of the model's parameters) and linearly increasing the computational cost of the soft-max operation. One option is to use individual characters or bytes to reduce the vocabulary size, but this increases the number of tokens, and context size has a $n^2$ cost for compute and memory in self-attention layers, the main bottleneck in transformer models. With this apparent trade-off between vocabulary size and token length, certain vocabulary-learning algorithms that seem to strike this balance became popular. For example, BPE and WordPiece \addcites. \writemore. An important distinction with these approaches is the difference between their vocabulary-learning algorithm and their inference algorithm, \writemore. Formally, \writemore. 

Both the inference and vocabulary-learning algorithms typically operate after text has already been split into word-like units; with what was previously called `tokenization' in the classical pipeline now referred to as ``pre-tokenization''. 

Several of the steps previously performed separately by the classical pipeline are now packaged together into a single `tokenizer' by libraries such as Huggingface Tokenizers.\footnote{\href{https://huggingface.co/docs/tokenizers/index}{huggingface.co/docs/tokenizers/}} This tokenisation pipeline is shown in \cref{fig:12-pipelinecomparison}. First, \textbf{normalization} performs the text-cleaning steps, before text is split into word-like units via \textbf{pre-tokenization}. The BPE pre-tokenizer not only splits text into word units, but also converts each unit into a byte-based representation, ensuring that the initial vocabulary contains only 256 items instead of every unique character in UTF-32. The tokenizer \textbf{model} then converts each pre-token into one or more tokens using a particular inference algorithm, such as longest-prefix matching for WordPiece or deterministic-merge application by BPE. These tokens are mapped to unique IDs to facilitate lookup into a LM's embedding layer. The pipeline may also include \textbf{post-processing} to add special tokens and \textbf{decoding} to convert IDs back into text for text generation tasks. 

The design and availability of this tokenization pipeline has largely been driven by the sheer scaling capabilities of LM architectures, the largest of which are called ``large'' language models (LLMs). Now, the vast majority of language models are distributed on platforms such as Huggingface with an associated tokenizers that consistently process orthographic text, pre-tokenize the text to preserve word boundaries, and return tokens representing subwords.

Packaging up these pre-processing steps into a single tokenizer provides convenience, but this has had consequences. In classical NLP, it was an important step to carefully considering the data cleaning operations. For example, removing punctuation... \writemore. The scaling ability of modern LMs has largely shifted the focus to language model architectures and machine learning algorithms, with many models using default tokenisers from Huggingface without considering the impact. This is particularly the case for studies using smaller LMs to study language or acquisition (see \cref{sec:12-babylm}) where subwords in particular may not be an appropriate base unit, nor the orthographic domain if simulating spoken speech. In these cases the default representation is no longer as crucial for performance, yet the convenience of the existing frameworks facilitates its use. 

\Zeb{Here it might be good to try to formalise ``input representation'' with the many axes that you can compare them on. Perhaps a figure showing where classical and modern NLP fall on those axes, along with phonemic and other alternatives.}

The following sections provide further background and related work concerning the input representations used in language modelling. First, \cref{sec:12-default} provides a definition of the \textbf{default input representation} and an overview of tokenisation studies that have criticised, analysed and suggested alternative tokenisers that all produce tokens following this representation. \Cref{sec:12-phonemic} presents the \textbf{phonemic input representation} in contrast, an under-studied alternative representation and the subject of this thesis. Finally, these two representations are not the only possibilities; alternatives are briefly described in \cref{sec:12-alternatives}.

\Zeb{Some statement here about potential of phoneme tokenisation and the fact that alternative input representations are largely under-studied in NLP, due to the convenience of the default one. Maybe here go into the historical use of the input representation and that it has been vastly understudied. }. 

\subsection{The default input representation}\label{sec:12-default}

\Zeb{Give background on what LMs use by default. Define as subwords + orthographic + pre-tokenization.}

\Zeb{Define tokenizers more formally and discuss BPE and WordPiece etc and pointing to some surveys about these methods. Greed is all you need method.}

\Zeb{Plausibility of subwords, issues in certain languages, misalignment etc. Superword paper as an alternative to word boundary-based pretokenization. Morpheme-aligned alternatives that never caught on. Many alternative approaches. Character-level models and byte-level models. Attempts to go `token-free' but byte-level still an arbitrary token choice.}

\zeb{Change this to be about the tokens rather than the tokenizers}
\begin{enumerate}
    \item process written (orthographic) text,
    \item pre-tokenize the text to preserve word boundaries, and
    \item split pre-tokenized into tokens that represent subwords.
\end{enumerate}

The combination of these three features will henceforth define the \textbf{default input representation}. Now, the vast majority of language models are distributed with an associated tokenizer. The tokenizer converts noisy text to unique token IDs, which are fed through the model, which produces contextual embeddings. Auto-regressive LMs are trained with a next-token prediction objective, allowing them to generate text one token at a time, which the tokenizer can convert back into readable text. Alternatively, the contextual representations are directly used, or the model is fine-tuned on a downstream task involving labelled data. There are countless variations to this setup in modern NLP but the vast majority use tokenisers...

In recent years, tokenization has become an increasingly popular topic, with the default configurations of popular tokenizers often critiqued and analysed, and many studies proposing improvements and alternatives to existing tokenizers. An overview of this work is provided in \cref{sec:12-default}. Despite this scrutiny, the focus is still on tokenizers that produce the default input representation. 

\subsection{The phonemic input representation}\label{sec:12-phonemic}

\Zeb{Formal definition of what I mean by the phonemic input representation. This is where the three transformations are described. Similar to character-level. Maybe here worth going more into detail with character-level models and tabula rasa model.}

%\subsubsection{Practical uses of the phonemic input representation}

\Zeb{Survey of methods using phonemes. Briefly discuss how important phonemes were for speech recognition technology but now a lot of that is end-to-end. Discuss more recent uses in LMs. End by stating maybe we're resource limited, as discussed in \cref{chapter:resources}}

\begin{figure}[t]
    \centering
    \includegraphics[width=0.99\linewidth]{example-image-b}
    \caption{The phrase ``example phrase'' encoded using the default input representation compared to the phonemic input representation.}
    \label{fig:12-representation}
\end{figure}

An alternative to the default input representation, and the topic of this this thesis, is the \textbf{phonemic input representation}. This is defined in contrast to above as:

\begin{enumerate}
    \item process phonemic text,
    \item do not preserve word boundaries, and
    \item split text into tokens that represent individual phonemes.
\end{enumerate}

A major use of the phoneme representation is phonological experimentation. This includes connectionist models of language processing, computational models of word segmentation and word-level phonotactic models, as discussed in \cref{sec:12-phoneval}. A practical use was speech recognition technology. Was important to create phoneme-level models along with language models in a complicated system to do TTS, STT, voice recognition, language recognition, etc. Many resources were built to support this, many phonological datasets. A detailed analysis of existing phonological resources in given in \cref{chapter:resources}.

Despite this history, less work exploring phoneme representation in modern NLP landscape. One reason could be that the scaling abilities of LLMs have favoured text, which is more readily online and regularly scraped. Speech technology has also largely moved to end-to-end directly from/to audio, without needing discrete phoneme models.

Yet, phoneme LMs do offer some practical benefits which have been explored, \writemore. There are also analytical benefits, as discussed in \cref{sec:12-phoneval}.

% One reason could be a lack of resources, which is the topic of \cref{chapter:resources}. A benefit of exploring alternative input representations is to ablate the effect of each of the features of the default representation, as explored in \cref{chapter:modelling}. The cognitive aspect... \cref{chapter:phonology}... and could improve tokenisation methods... \cref{chapter:infotokenization}.

Phoneme language models. Many word-level models to study phonology. Very few examples of running text. Some examples of working with audio directly, as discussed in the next section.

Phoneme representation could be compared against default representation along three axes. There is some work comparing these effects. Several studies have challenged (3) by demonstrating that character-based or byte-based models can still be effectively utilised in language models \addcites. These argue to be "token free" but according to \addcites is still an arbitrary subword representation. A few studies have challenged (2) by relaxing the word-boundary pre-tokenization constraint to produce `superword' tokens \addcites or explore the effect of removing whitespace altogether to achieve a `tabula rasa' input representation \addcites but still orthographic text. Finally, Bunzeck paper. \cref{chapter:modelling} properly compares these to establish the effect of the phoneme representation in modern language models.

\subsection{Alternative input representations}\label{sec:12-alternatives}

\Zeb{Briefly survey speech models, pixel tokenisation etc. but probably don't go into too much detail.  Maybe patches work goes here as well.}

\section{Phonological experimentation using language models}\label{sec:12-phoneval}

\Zeb{Phoneme representation has been used in various contexts. First three sections below are all examples of trying to reach some conclusion about language by using a language model's predictions to justify some kind of argument. Final section is more about language models themselves, what kind of linguistic knowledge they acquire.}

\subsection{Connectionist models of language processing}

\Zeb{Brief survey of old connectionist models looking at language processing}

\subsection{Computational models of word segmentation}

\Zeb{Discuss word segmentation in detail here and discuss its relationship with subword tokenisation, forward referencing \cref{chapter:infotokenization}.}

\subsection{Cross-lingual studies of phonology}

\Zeb{More recent studies that use LMs to study phonology by using tiny LMs that train just over word types rather than tokens in context. Conclude that there have been few attempts at training cross-lingual LMs on naturalistic running text to get a model of phonology for each language.}

\subsection{Phonological evaluation of language models}

\Zeb{Discuss prior evaluation used to evaluate language models for phonological knowledge. Possibly start with wider background looking at how people evaluate LMs for knowledge of linguistic structure. Discuss phonologybench as the opposite of what we're looking for. BabySLM is more relevant, discussed in the next section.}

\section{Pre-training on developmentally-plausible data}\label{sec:12-babylm}

\Zeb{Idea that ``simulations must closely emulate real-life situations by training on developmentally plausible corpora'' in order to gain insights both about what language models can learn, and improve understanding about how infants learn language. Clearly could be valuable insights for phonology in this area.}

\Zeb{Early example is BabyBERTa paper, who trained on a version of AOCHILDES. Briefly discuss findings and criticisms.}

\Zeb{Later, BabyLM challenge created a framework for training and evaluting such models. Still motivated by developmentally plausible data, but still using default input representation. Findings more about architectures for low-data than specifically for insights into human learning. This framework provides a good way of testing this size model.}

\Zeb{Finally, BabySLM paper, which focused more on speech models. They also trained models on portions of CHILDES, including AOCHILDES, as well as Seedlings, an audio dataset, and compare speech, phonemes and text as input representations.}

\Zeb{Besides BabySLM and Bunzeck papers, very few examples of phoneme-based training on developmentally-plausible data, possible due to resource limitations (see \cref{chapter:resources}). However, these papers provide a useful starting point for establishing whether phoneme-based training is plausible and the datasets and evaluation criteria described below are leveraged in this thesis..}

\Zeb{Should mention \citet{huebner_structured_2018} as an early example of modelling child-directed speech using SRNs, LSTMS and skip-gram.}

\Zeb{Review \citet{wilcox_bigger_2025} and Suchir's paper for wider position pieces on the role of babylm for wider research.}

\Zeb{BabyLM framework is appropriate for two reasons. Firstly, the scale: it's appopriate for academic budgets, phoneme models can be useful for low-resource, and phoneme data is too limited for larger scale models (see next chapter). Secondly, the developmental/analytical angle. These models have been branded as useful for cognitive research due to the developmental plausibility. Phoneme representation so far has mostly been used for analytical purposes, so using develpmentally plausible framework is logical - and using the phoneme representation provide an additional aspect of plausibility beyond written text and exploration is worthwhile.}

\subsection{Datasets}

\subsubsection{The CHILDES database}

\Zeb{Describe CHILDES. Discuss people who have done language modeling with CHILDES (e.g. BabyBERTa).}

The Child Language Data Exchange System (CHILDES) is a repository of child-centred data originally developed with the aim of preserving and standardising data used for child language development research \citep{macwhinney1985child}. The project later grew into TalkBank, which contains over 1.4TB of transcript data and 5TB of associated media data across several ``banks'' \citep{macwhinney_understanding_2019}. Each bank focuses on a different area of human communication, with a general focus on spoken communication, with the CHILDES banks now containing child-caregiver interactions in over 40 different languages thanks to the efforts of hundreds of contributors over the last 40 years. 

CHILDES is an extremely valuable resource for research on child language development and has led to many thousands of published articles since its release. Due to the commitment to open-data sharing, the influence of these resources has also extended to other fields of research. For instance, in a recent assessment on the impact of the TalkBank project, \citet{bernstein_ratner_augmenting_2024} noted that a corpus she had contributed to CHILDES --- originally collected to study the acoustic features of child-directed speech \citep{Ratner_1984} --- was still leading to new insights forty years later:
\zeb{Probaby worth mentioning that this became the BR corpus for word segmenation.}
\begin{quote}
The corpus has been further used to test models of unsupervised induction of grammar in machine language learning \citep{glushchenko_programmatic_2019}, a prospect not remotely envisioned during the original study, when data were collected on reel-to-reel analog tapes, and acoustically analyzed using a dedicated mainframe computer that had to be booted with punched paper tape.
\end{quote}

For researchers seeking developmentally-plausible corpora to use as pre-training data for language models, CHILDES is a natural fit. For instance, BabyBERTa was pre-trained on the AO-CHILDES corpus (Age-Ordered CHILDES, \citep{huebner2021using}). AO-CHILDES contains approximately 5M words from the North American English sections of CHILDES. Specifically, it contains all utterances involving children aged 0 to 6, sorted by child age, with non-adult speech removed (thus simulating the theoretical input received by a child). In their study, \citet{huebner-etal-2021-babyberta} were interested in the outcome of pre-training BabyBERTa on AO-CHILDES compared to Wikipedia (which was considered a more representative dataset for NLP at the time). Using a linguistic benchmark based on BLiMP (see \cref{sec:12-blimp} below), they found that BabyBERTa achieved a higher accuracy when trained on AO-CHILDES, that the choice of corpora had an effect across the pre-training corpus and that ordering multiple corpora by grammatical complexity could `scaffold' learning. These findings validated the importance of using developmentally-plausible corpora for BabyLM research and the scaffolding results were a precursor of many of the curriculum learning approaches taken in the first edition of the BabyLM challenge \addcites.

\Zeb{Insert other examples of language modelling with CHILDES}

\Zeb{Insert a few phoneme examples of language modelling with CHILDES, discuss PhonBank}

\subsubsection{The BabyLM dataset}

\Zeb{BabyLM dataset description. Note that they wanted to reach 100M words, which they couldn't just do with CHILDES. Can slightly criticise, not quite developmentally plausible but more so than web-scraped corpora (maybe show figure comparing CHILDES, BabyLM and Pile). Maybe briefly mention other cognitively plausible datasets that have sprung up recently, like German BabyLM, KidLM, storybooks, chat-gpt generated data, future BabyLM multilingual.}

\subsection{Evaluation metrics}

Below are a description of the main benchmarks used in this thesis to evaluate LMs. 

\subsubsection{BLiMP}\label{sec:12-blimp}

\subsubsection{GLUE}

\subsubsection{BabySLM}

\section{Summary}

\Zeb{Go back to research questions. Need to establish whether resources exist. Need to do a thorough comparison of input representations and determine how to do language modeling with phonemes. Need to see what insights can be gained from phonological experimentation with such models. BabyLM framework gives good start to use for this stuff, but still limited ways to evaluate phoneme LMs.}