\appendix
\chapter{Extra Information}
Some more text ...

\section{Corpus Phonemizer tool usage}
\label{appendix:corpus-phonemizer-usage}

\lstset{basicstyle=\small\ttfamily}
\begin{lstlisting}[float=tp, breaklines=true,caption={An extract taken from the help menu of \corpusphonemizer, displaying the tool's usage.}]

usage: corpus_phonemizer.py [-h] [-k] [-v] [-u] [-i INPUT_FILE]
       [-o OUTPUT_FILE] {epitran,phonemizer,pingyam,pinyin_to_ipa} language

Phonemize utterances using a specified backend and language.

positional arguments:
  {epitran,phonemizer,pingyam,pinyin_to_ipa}
                                        The backend to use for phonemization.
  language                              The language to phonemize.

options:
  -h, --help                            show this help message and exit
  -k, --keep-word-boundaries            Keep word boundaries in the output.
  -v, --verbose                         Print debug information.
  -u, --uncorrected                     Use the wrapper's output without
                                        applying a folding dictionary to correct
                                        the phoneme sets.
  -i INPUT_FILE, --input-file INPUT_FILE
                                        Input file containing utterances (one
                                        per line). If not specified, reads from
                                        stdin.
  -o OUTPUT_FILE, --output-file OUTPUT_FILE
                                        Output file for phonemized utterances.
                                        If not specified, writes to stdout.

Example usage:
  python corpus_phonemizer.py epitran --language eng-Latn --keep-word-boundaries --verbose < input.txt > output.txt
\end{lstlisting}

\section{Corpus Phonemizer Folding Dictionaries}
\label{appendix:folding-dictionaries}